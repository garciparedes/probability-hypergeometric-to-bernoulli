% !TEX root = ./article.tex

\documentclass{article}

\usepackage{mystyle}
\usepackage{myvars}

%-----------------------------

\begin{document}

	\maketitle
  \thispagestyle{empty}

%-----------------------------
%	TEXT
%-----------------------------

  \section{Distribución Hipergeométrica}
  \label{sec:description}

    \paragraph{}
    La distribución hipergeométrica $X \sim H(n, N, D)$ se corresponde con una distribución de probabilidad para variables aleatorias discretas. Esta modeliza la situación de obtener $k$ casos favorables, de $n$ observaciones, sobre una población de $N$ individuos, donde $D$ representan la situación de éxito y $N-D$ la de fracaso. Por tanto, modeliza la misma situación que la distribución binomial, solo que en este caso se presupone que cada observación se realiza \textbf{sin reemplazamiento}, mientras que en la distribución binomial se asume reemplazamiento.

    \paragraph{}
    En las siguientes secciones se demuestra que una distribución hipergeométrica $X \sim H(n, N, D)$ es equivalente a la suma de $n$ distribuciones de bernoulli dependientes entre sí. Por tanto, $X = X_1 + ... + X_i + ... + X_n$ donde $X_i \sim B(\frac{D}{N})$ representa la $i$-ésima observación.

  \section{Demostración para $n=2$}
  \label{sec:demostration_1}

    \paragraph{}
    A continuación se demuestra el caso para dos variables:

    \begin{align}
      X   &\sim H(2, N, D)    &             &\\
      X   &= X_1 + X_2        &             &\\
      X_1 &\sim B(\frac{D}{N})& P[X_1 = 1]  &= \frac{D}{N}\\
      X_2 &\sim B(\frac{D}{N})& P[X_2 = 1]  &= P[X_2 = 1 \land X_1 = 0] + P[X_2 = 1 \land X_1 = 1] \\
          &                   &             &= P[X_2 = 1 \mid X_1 = 0]*P[X_1 = 0] + P[X_2 = 1 \mid X_1 = 1]*P[X_1 = 1] \\
          &                   &             &= \frac{D}{N-1}*(1-\frac{D}{N}) + \frac{D-1}{N-1}*\frac{D}{N} \\
          &                   &             &= \frac{D}{N}
    \end{align}

  \section{Demostración para $n=3$}
  \label{sec:demostration_1}

  \paragraph{}
  A continuación se demuestra el caso para dos variables:

  \begin{align}
    X   &\sim H(3, N, D)    &             &\\
    X   &= X_1 + X_2 + X_3  &             &\\
    X_1 &\sim B(\frac{D}{N})& P[X_1 = 1]  &= \frac{D}{N}\\
    X_2 &\sim B(\frac{D}{N})& P[X_2 = 1]  &= P[X_2 = 1 \land X_1 = 0] + P[X_2 = 1 \land X_1 = 1] \\
        &                   &             &= P[X_2 = 1 \mid X_1 = 0]*P[X_1 = 0] + P[X_2 = 1 \mid X_1 = 1]*P[X_1 = 1] \\
        &                   &             &= \frac{D}{N-1}*(1-\frac{D}{N}) + \frac{D-1}{N-1}*\frac{D}{N} \\
        &                   &             &= \frac{D}{N} \\
    X_3 &\sim B(\frac{D}{N})& P[X_3 = 1]  &= ... \\
        &                   &             &= ... \\
        &                   &             &= \frac{D}{N} \\
  \end{align}

  \section{Demostración caso general}
  \label{sec:demostration_1}

    \paragraph{}
    [TODO ]

%-----------------------------
%	Bibliographic references
%-----------------------------
	\nocite{muest2017}

  \bibliographystyle{alpha}
  \bibliography{bib}

\end{document}
